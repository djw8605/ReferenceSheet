\documentclass[7pt]{article}

\usepackage{amsmath}
\usepackage[margin=.25in, landscape]{geometry}
\usepackage[ruled, vlined]{algorithm2e}
%\usepackage{algorithmic}
\usepackage{multicol}

\linespread{1}
\begin{document}

\begin{multicols*}{4}
\subsection*{Automata}
\subsubsection*{Regular languages}
DFA requires every state to have every possible transition.  NFA only requires you have the transitions necessary to accept.

\subsubsection*{Pumping Lemma for regular languages}
If $A$ is a regular language, then there is number $p$ (the pumping length) where, if $s$ is any string in $A$ of at least $p$, then $s$ may be divided into three pieces, $s = xyz$, satisfying the following conditions:
\begin{enumerate}
\item for each $i \geq 0$, $xy^iz \in A$
\item $|y| > 0$, and
\item $|xy| \leq p$
\end{enumerate}

Prove pumping lemma by contradiction.  Assume the language is regular, then find an instance where $xyyz$ is not in the language.

\subsubsection*{Decidability}
All Context Free Grammars are decidable.  For regular languages (DFA, NFA, Regex), construct a Turing Machine that decides yes or no.

\noindent
On input $A$ where $A$ is a DFA:
\begin{enumerate}
\item Mark the start state of $A$
\item Repeat until no new states get marked:
\begin{itemize}
\item Mark any state that has a transition coming into it from any state that is already marked.
\end{itemize}
\item If no accept state is marked, accept; otherwise, reject.
\end{enumerate}

\subsubsection*{Reducibility}
Used to prove that a language is undecidable by showing that you can solve $A_{TM}$ (known to be undecidable) if the given language is decidable.  Key idea is to show that $A_{TM}$ is reducible to the language given.  IE, use the language given to show that you can use it to solve $A_{TM}$.

\subsection*{Algorithms}
\subsubsection*{NP-Completeness}
{\bf Show a problem is in NPC}
\begin{itemize}
\item Show it's in NP by showing it's verifiable in poly time using
  some certificate
\item Reduce \emph{from} any known NPC problem (all possible instances
  of one) to the unknown NPC problem in poly time.
\end{itemize}

{\bf NPC problems to reduce from}
\begin{itemize}
\item HAM-CYCLE: Does a graph $G$ have a Ham-cycle (a simple cycle
  that visits every vertex exactly once)?
\item CIRCUIT-SAT: Is there a satisfying assignment of inputs to a
  circuit that makes it output 1?
\item SAT: Does some boolean formula have a set of satisfying assignments?
\item 3-CNF-SAT: Ditto, but the SAT takes on the specific form of
  $(x_k \wedge x_j \wedge x_i) \vee (x_k \wedge x_v \wedge x_l) ...$
\item CLIQUE: Does graph $G$ contain a clique of size $k$?
\item VERTEX-COVER: Does graph $G$ have a set of $k$ vertices that
  touch every edge
\item TSP: Does a complete, weighted graph $G$ have a HamCycle of
  total weight $\leq k$?
\item SUBSET-SUM: Is there a subset of integers from $S$ that some to
  exactly $t$? 
\end{itemize}

    {\bf Reductions}:
    \begin{itemize}
    \item Circuit-Sat $\rightarrow$ SAT: For each input to the CS, create a
      variable in the SAT.  For each output of a logic gate, create a
      variable.  Clauses in the SAT are the ouput iff (input gate
      input gate ...). SAT is conjunction of clauses (including final
      output).     
    \item SAT $\rightarrow$ 3-CNF-SAT: Build a parse tree (outermost
      to innermost).  This gets it to at most 3 elements per clause.
      Then generate a truth table for the 3 elements, grab the ones
      that evaluate to 1, or them, use DeMorgan's.  
    \item 3-CNF-SAT $\rightarrow$ Clique: For each clause, make 3
      vertices (representing the variables in the clause).  Draw edges
      between vertices if the vertices aren't negations of each other
      and they aren't from the same clause.
    \item Clique $\rightarrow$ Vertex-Cover: Find the complement of
      the graph in the Clique problem.  $VC-k$ becomes $|V| - clique-k$
    \end{itemize}


\subsection*{Dynamic Programming}
Basic idea: Decompose a problem into subproblems, resuing overlapping
solutions to the subproblems.

  \end{multicols*}
 
\end{document}